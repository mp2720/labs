\documentclass[a4paper, twocolumn]{article}
\usepackage[fontsize=13.2pt]{fontsize}
\usepackage{geometry}
\usepackage[T1, T2A]{fontenc}
\usepackage[utf8]{inputenc}
\usepackage[russian]{babel}
\usepackage{multirow}
\usepackage{amsmath}
\usepackage{titleps}
\usepackage{graphicx}
\usepackage[letterspace=200]{microtype}
\usepackage{array}
\usepackage{enumerate}
\usepackage[symbol]{footmisc}

\geometry{
    a4paper,
    total={170mm,257mm},
    left=20mm,
    top=20mm,
}

\newpagestyle{pagenumberfooterright}{
  \setfoot{}{}{\thepage}
}
\pagestyle{pagenumberfooterright}
\setcounter{page}{57}

\begin{document}

\newcommand{\QL}{\guillemotleft{}}
\newcommand{\QR}{\guillemotright{}}
\newcommand{\GQ}[1]{\guillemotleft{}#1\guillemotright{}}
\newcommand{\CellAlignCenter}[2]{\multicolumn{1}{#1}{#2}}
\newcommand{\xrowht}[2][0]{\addstackgap[.5\dimexpr#2\relax]{\vphantom{#1}}}

\DefineFNsymbols*{asterisks}{*{**}{***}}
\setfnsymbol{asterisks}
\let\origthefootnote\thefootnote
\renewcommand{\thefootnote}{\origthefootnote)}

жество близнецов. Наибольшей из известных пар-близнецов является пара (10000000009649, 10000000009651).

\textbf{3.}
Коснёмся ещё одного вопроса, связанного с простыми числами. Возьмём отрезок ряда натуральных чисел от 1 до не которого n включительно. На этом отрезке имеется определённое количество простых чисел. Число их принято обозначать через $\pi(n)$ \footnote{$\pi$ --- греческая буква \GQ{пи}, $n$ --- латинская буква \GQ{эн}, $\pi(n)$ --- читаeтся так: \GQ{пи от эн}.}. Чему равно $\pi(n)$ для отдельных значений $n$.

Для малых $n$ это легко подсчитать. Например $\pi(1) = 0, \pi(2) = 1, \pi(3) = 2, \pi(4) = 2, \pi(5) = 3, \pi(6) = 3, \pi(7) = 4, \pi(8) = 4, \pi(9) = 4, \pi(10) = 4$.

Сразу замечаетсся нерегулярное изменение $\pi(n)$. Вообще, никакой простой формулы для $\pi(n)$ написать нельзя.

Тем не менее, увеличивая $n$, можно заметить, что \GQ{средняя плотность} простых чисел, то есть отношение $\pi(n):n$ становится все меньше и меньше. Это хорошо видно из следующей таблицы:

\vspace{5mm}

\noindent
\scalebox{0.85}{
\setlength{\tabcolsep}{10pt}
\begin{tabular}{|r|r|l|}
    \hline
    & & \\[-10mmpt]
    % \\
    % \vspace{5mm}
    % \\
    \CellAlignCenter{|c|}{$n$} &
    \CellAlignCenter{c|}{$\pi(n)$} &
    \CellAlignCenter{c|}{$\pi(n):n$} \\
    & & \\
    \hline
    10 & 4 & 0,4 \\
    100 & 25 & 0,4 \\
    1 000 & 168 & 0,17 \\
    10 000 & 1 229 & 0,12 \\
    100 000 & 9 592 & 0,096 \\
    1 000 000 & 78 498 & 0,078 \\
    10 000 000 & 664 579 & 0,066 \\
    100 000 000 & 5 761 455 & 0,057 \\
    1 000 000 000 & 50 847 478 & 0,051 \\
    \hline
\end{tabular}
}

\vspace{5mm}

Доказано, что с возрастанием $n$ отношение $\pi(n):n$ приближается к нулю. Впервые этот факт доказал Леонард Эйлер --- величайший математик XVIII века. В дальнейшем великий русский математик Пафнутий Львович Чебышев уточнил результат Эйлера, доказав более общую теорему (см. \GQ{Квант} №5 за 1971 год, стр 1 --- 3), но об этом мы уже рассказывать не будем.

\end{document}